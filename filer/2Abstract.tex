\chapter{Abstract}
This report for a bachelor project in electronic engineering at Aarhus School of Engineering, Aarhus University presents an explanation of the groups bachelor project: "Charcot-foot design and development of temperature measurement device". The purpose of the original project is to design and develop a system that can measure the temperature in different places on the foot of a patient suffering from charcot-foot. The modified version of the project entails designing a system with several sensors connected to a power line with overlaid communication. The focus is on developing the technology behind the power line communication bus.\\
A thorough documentation of the whole process, covering requirement specification, system architecture, design and implementation has been made. A series of internal tests are used as a baseline for improving the system.\\
The group has approached the project in an iterative manner in order to develop and improve the system. A SCRUM-like board has been used to structure the tasks and a time schedule was made to structure the different phases in the project.\\
A fully working proof of concept system has been made consisting of a central data unit and two sensor nodes. They communicate over the custom power line communication bus developed in this project. A chapter detailing the further development needed is included in this report.\\
The project has netted both group members a better understanding of hardware development as a whole.\\

\chapter{Resume}
Denne rapport for et elektro bachelorproject på Ingeniørhøjskolen Aarhus, Aarhus Universistet præsenterer en gennemgang af gruppens bachelor project: "Charcot-fod design og udvikling af temperature-måle-device". Formålet med det originale projekt er at designe og udvikle et system der kan måle temperaturen på forskellige områder af foden på en patient der lider af charcot-fod. Det modificerede projekt omhandler designet af et system med flere sensorer forbundet til en forsyningslinje hvorpå der også er kommunikation. Fokus er på at udvikle teknologien bag "power line communication bus", forsyningslinjekommunikationsbussen.\\
Der er blevet udfærdiget grundig dokumentation gennem hele forløbet der dækker over kravspecifikation, systemarkitektur samt design og implementering. En række af interne tests har lagt til grundlag for at forbedre system.\\
Gruppen har brugt en iterativ fremgangsmåde til at udvikle og forbedre systemet. En SCRUM-lignende tavle har været brugt til at planlægge og strukturere opgaver og en tidsplan har været brugt til at strukturere forskellige faser af projektet.\\
Et virkende "proof of concept" system bestående af en central data enhed (CDU) og to sensor noder er blevet lavet. Enhederne kommunikere ved hjælp af  forsyningslinjekommunikationsbussen, som der er udviklet i dette projekt. Et kapitel der omhandler den videre udvikling af projektet er også forfattet i denne rapport.\\
Projektet har resulteret i at begge gruppemedlemmer har fået en bedre forståelse for udvikling af hardware.