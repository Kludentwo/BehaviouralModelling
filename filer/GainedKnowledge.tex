\chapter{Gained Knowledge}
Reading The Zynq Book provides a good insight into the Zynq platform and how to use it. The knowledge is nice to have but not a necessity in order to work with the platform. The multitude of user guides and workshops provides the basis knowledge needed in order to develop with the platform. The workshops are a neat way to get acquainted with the different tools in the vivado design suite with their step by step guidance with good images. \\

Ultimately the only way to get better at designing using High-Level synthesis is to work on a design project. Working on the project has given me the knowledge needed to design and implement high level code and synthesising it to lower level code. It has also broaden my knowledge in the field of the possibilities with FPGA technologies. \\
Building upon my basic knowledge of VHDL, I have learned about how to optimise your design with various optimisation directives such as pipelining and allocation. These directives can be used to fine tune a design, such that it can meet the requirements set by the project. \\

When looking for research papers on High-Level synthesis, it becomes apparent that the field is not very mature yet. Papers explain the evolution of High-Level synthesis\cite{martin2009high} or how to develop with High-Level Synthesis\cite{cong2011high}. There are few detailing on how to implement a least square problem solver using FPGA\cite{yang2009fpga} but this is not using High-Level Synthesis and the neat optimisation directives the follows. Comparisons are done with respect to computers CPUs. Closer related papers use other metrics making comparing their results to mine difficult\cite{skalicky2014performance}.\\

Implementing my least square problem solver using High-Level synthesis was made easier by the tool and took far less time, than if I were to implemented in VHDL directly. The tool is very helpful and easy to work with and the environment enables very fast development. Since the languages used are derived from higher level languages like C++, High-Level synthesis is very relevant when developing systems with speed or area constraints. Most C, C++ and SystemC projects can be done using the tool. The technology is ready to be more widely used when research and development companies start adopting the system.